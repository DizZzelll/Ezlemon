\documentclass{beamer}
\usepackage[english,russian]{babel}
\usepackage[utf8]{inputenc}
% Стиль презентации
\usetheme{Warsaw}
\begin{document}
\title{Билет №2}  
\author{Лоскутов Дмитрий Андреевич}
\institute{СПбГЭТУ «ЛЭТИ»}
\date{Санкт-Петербург, 2019} 
% Создание заглавной страницы
\frame{\titlepage} 
% Автоматическая генерация содержания
\begin{frame}{Содержание}
 \begin{thebibliography}{10}
\beamertemplatebookbibitems
\bibitem{1}
{\sc 1}, {\em Информационная культура и информационная безопасность личности.}.
\bibitem{2}
{\sc 2}, {\em Внутренние сети с возможностями сети internet - сети intranet. Локальная сеть университета: основные возможности,архитектура с точки зрения пользователя, режимы использования.Информационные ресурсы сети.Права и обязанности пользователей в сети.}.
\end{thebibliography}
\end{frame}

\begin{frame}{Информационная культура}
   Информационная культура - это конкретный уровень формирования информационных процессов, уровень создания, сбора, переработки и хранения информации, степень удовлетворения в определенной мере потребностей человека в информационном общении
\end{frame}

\begin{frame}{Информационная культура}
 Специалисты называют следующие признаки информационной культуры человека:
 
1. умение адекватно выражать свою потребность в конкретной информации;\\
2. способность перерабатывать полученную информацию и создавать новую;\\
3. эффективно осуществлять поиск необходимых данных;\\
4. умение вести индивидуальные поисковые информационные системы;\\
5. способность адекватно оценивать информацию;\\
6. умение правильно отбирать необходимые данные;\\
7. способность к компьютерной грамотности и информационному общению.

\end{frame}

\begin{frame}{Информационная культура}
    Информационная культура совершенно не  сводится к разрозненным знаниям и умениям работы за компьютером. Она предполагает информативную направленность целостной личности, которая обладает мотивацией к применению и усвоению новых данных. Информационная культура, по мнению специалистов, рассматривается как одна из граней личностного развития. Это путь универсализации качеств человека. 
\end{frame}

\begin{frame}{Информационная безопасность личности}
    Информационная безопасность личности – это состояние и условия жизнедеятельности личности, при которых реализуются ее информационные права и свободы
\end{frame}

\begin{frame}{Информационная безопасность личности}
    Жизненно важные интересы личности в информационной сфере следующие\\
1. Соблюдение и реализация конституционных прав на поиск, получение прав и
распространение информации.\\
2. Реализация прав гражданина на неприкосновенность частной жизни.\\
Угрозы интересам личности в информационной сфере.\\
1. Применение нормативно-правовых актов, противоречащих конституционным правам граждан.\\
2. Противодействие, в том числе со стороны криминальных структур, реализация гражданами прав на неприкосновенность частной жизни.
\end{frame}

\begin{frame}{сети Intranet}
    интранет — это «частный» Интернет, ограниченный виртуальным пространством отдельно взятой организации. Intranet допускает использование публичных каналов связи, входящих в Internet, (VPN), но при этом обеспечивается защита передаваемых данных и меры по пресечению проникновения извне на корпоративные узлы. 
\end{frame}


\begin{frame}{Информационные ресурсы сети}
    Информационные ресурсы, доступные через Интернет, огромны. Это десятки миллионов документов, представленных различными способами, число которых постоянно увеличивается. В зависимости от способа представления, вида и характера информации разнятся и методы доступа к ней, поэтому, прежде чем рассматривать методы поиска, рассмотрим классификацию информационных ресурсов.\\
    
    По принципу организации и использования средства поиска можно разделить на каталоги (справочники, директории) и поисковые машины  
\end{frame}


\begin{frame}{Права и обязанности пользователей сети}
Пользователь информации имеет право:\\
Получать, распространять, предоставлять информацию;\\
Использовать информационные технологии, информационные системы и информационные сети;\\
Знакомиться со своими персональными данными;\\
\end{frame}

\begin{frame}{Права и обязанности пользователей сети}
Пользователь информации обязан:\\
Cоблюдать права и законные интересы других лиц при использовании информационных технологий, информационных систем и информационных сетей;\\
Принимать меры по защите информации;\\
Обеспечивать сохранность информации, распространение, предоставление которой ограничено, и не передавать ее полностью или частично третьим лицам без согласия обладателя информации;\\
\end{frame}

\end{document}