\documentclass[14pt]{extreport}
\usepackage[utf8]{inputenc}
\usepackage[english, russian]{babel} 
\usepackage{mathtext}  
\usepackage{amsmath,amssymb,amsfonts,amsthm}
\usepackage{indentfirst}
\usepackage{tikz}
\usepackage{pgfplots}
\pgfplotsset{compat=1.9}
\usepackage{geometry}
\title{KURSACH XXX}
\author{los.dimasya 8871}
\date{December 2018}
\begin{document}
\maketitle
\centering МИНОБРНАУКИ РОССИИ
САНКТ-ПЕТЕРБУРГСКИЙ ГОСУДАРСТВЕННЫЙ
ЭЛЕКТРОТЕХНИЧЕСКИЙ УНИВЕРСИТЕТ 
«ЛЭТИ» ИМ. В.И. УЛЬЯНОВА (ЛЕНИНА)\\[1cm]

Кафедра РАПС\\[3cm]

КУРСОВАЯ РАБОТА (ЛАТЕКС)\\[0.2cm]
по дисциплине «Информейшон»\\[0.2cm]
Тема: Курсачик :3\\[5cm]

Студент гр. 8871 \hspace{4cm}		Лоскутов Д.А.\\[0.4cm]
Преподаватель \hspace{4cm}		Прокшин А.Н.\\[2cm]

Санкт-Петербург\\[0.5cm]
2018

\newpage

\chapter{Coderjanie}

\begin{enumerate}
\item\footnotesize {Даны функции $f(x) = \sqrt{3}sin(x) + cos(x)$ и $g(x) = cos(2x + \pi/3) - 1$\\
a)	Решить уравнение $f(x) = g(x)$.\\
b)	Исследовать функцию $h(x) = f(x) - g(x)$ на промежутке $[0 ; (5*pi)/6]$}\\
\item Найти коэффициенты кубического сплайна, интерполирующего данные, представленные в векторах Vx и Vy. \\
Построить на одном графике: функцию f(x) и функцию f1(x), получен-ную после нахождения коэффициентов кубического сплайна.\\
Представить графическое изображение результатов интерполяции ис-ходных данных различными методами с использованием встроенных функ-ций:\\
$cspline(Vx,Vy), pspline(Vx,Vy), lspline(Vx,Vy)$ и $interp(Vk,Vx,Vy,x)$.

\item Решить задачу оптимального распределения неоднородных ресурсов.\\
На предприятии постоянно возникают задачи определения оптималь-ного плана производства продукции при наличии конкретных ресурсов (сырья, полуфабрикатов, оборудования, финансов, рабочей силы и др.) или проблемы оптимизации распределения неоднородных ресурсов на произ-водстве. Рассмотрим несколько возможных примеров постановки таких за-дач.\\
\textbf{Постановка задачи В} (вариант 14). Пусть в распоряжении завода железобетонных изделий (ЖБИ) имеется m видов сырья (песок, щебень, цемент) в объемах аi. Требуется произвести продукцию n видов. Дана технологическая норма сij потребления отдельного i-го вида сырья для изготовления единицы продукции каждого j-гo вида. Известна прибыль Пj, получаемая от выпуска единицы продукции j-гo вида. Требуется определить, какую продукцию и в каком количестве должен производить завод ЖБИ, чтобы получить максимальную прибыль.
\end{enumerate}

\newpage

\textbf{Список обозначений и сокращений.\\[0.4cm]}

\textbf{x} - аргумент\\[0.4cm]

\textbf{f(x)} - функция f(x)\\[0.4cm]

\textbf{g(x)} - функция g(x)\\[0.4cm]

\textbf{h(x)} - функция h(x)\\[0.4cm]

\textbf{$\frac{d}{dx}*h(x)$} - первая производная функции h(x)\\[0.4cm]

\textbf{$\frac{d^2}{dx^2}*h(x)$} - вторая производная функции h(x)\\[0.4cm]

\textbf{Vx, Vy} - векторы данных\\[0.4cm]

\textbf{cspline, lspline, pspline} - функции, используемые для нахождения коэффициентов сплайна.\\[0.4cm]

\textbf{f1(x)} - линейная интерполяция данных, представленных в векторах Vx и Vy\\[0.4cm]

\textbf{П$_$j} - прибыль предприятия\\[0.4cm]

\textbf{X} - Количество изделий j-го наименования, которое может производить предприятие\\[0.4cm]

\textbf{X5, X6, X7} - фиктивные изделия, при изготовлении которых используют каждый оставшийся вид ресурса.\\[0.4cm]

\newpage

\chapter{Выполнение 1 задания}

\section{Решение уравнения f(x) = g(x)}

\fbox{$f(x) := \sqrt{3}sin(x) + cos(x)$}\\
\fbox{$g(x) := cos(2x + \pi/3) - 1$}\\[0.4cm]
\fbox{$f(x) = g(x)$}\\[1cm]
\fbox{$f(x) = g(x) solve,x -> \begin{bmatrix}
\frac{-2\pi}{3} + 2 \pi k \\
\frac{-\pi}{6} + 2 \pi k\\
\frac{5\pi}{6} + 2 \pi k  
\end{bmatrix} $}\\[2cm]

Уравнение имеет решение при $x =\frac{-2\pi}{3} + 2 \pi k$, $x=\frac{-\pi}{6} + 2 \pi k$ , $x =\frac{5\pi}{6} + 2 \pi k $  .
\newpage


\begin{tikzpicture}
\begin{axis}
\addplot gnuplot  {1.7320508075688772935274463415059*sin(x)+cos(x)};
\legend {F(x)}

\end{axis}
\end{tikzpicture}

\end{document}
